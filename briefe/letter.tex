% !TeX encoding     = utf8
% !TeX program      = pdflatex
% !TeX spellcheck   = de-DE

% ---------------------------
% Template: letter
% Author:   Felix Faltin
% Version:  1.0
% ---------------------------
% Erstellung eines einfachen Briefs ohne jegliche Einstellungen.
% Die Klasse letter berücksichtigt nicht die deutsche Formiertierung 
% von Briefe, sodass der Empfänger nicht zwingend an der richtigen
% Position steht. Die Klasse scrlttr2 ist vorzugsweise zu verwenden.
% ---------------------------

% ---------------------------
% Dokumentklasse
% ---------------------------
\documentclass[12pt,a4paper]{letter}

% ---------------------------
% Einbinden von Packages
% ---------------------------
\usepackage[utf8]{inputenc}
\usepackage[T1]{fontenc}
\usepackage[ngerman]{babel}

% ---------------------------
% Fakulative Packages
% ---------------------------
\usepackage{blindtext}

% ---------------------------
% Absender
% ---------------------------

% --- Adresse ---
\address{%
    Max Mustermann\\
    Musterstraße 1\\
    01234 Hanshausen}
   
% --- Signatur --- 
\signature{Max Mustermann} 

\begin{document} 
    \begin{letter}{%
            % --- Empfänger ---
            Maxi Musterfrau\\
            Musterweg xx\\
            01234 Mustersonderhausen
        } 
        
        % --- Ansprache ---
        \opening{Sehr geehrte Damen und Herren,} 

         % --- Inhalt ---
        \blindtext
     
        % --- Grußformel
        \closing{Mit freundlichen Grüßen}
        
        % --- Postskriptum ---
        %\ps{adding a postscript}

        % --- Anlage(n) ---
        \encl{%
            Hauseigene Briefformate\\
            Hinweise zu firmenspezifischen Anpassungen
        }
        
        % --- Verteiler ---
        %\cc{Cclist} 
    \end{letter} 
\end{document}