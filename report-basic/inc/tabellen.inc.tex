\chapter{Tabellen}
		
	\section{Einfache Tabellen ohne Pakete}
		% ---------------------------
		% Einfache Tabellen darstellen
		% ---------------------------
		% Spaltenangabe durch
		%   l 		- links
		%   c 		- zentriert
		%   r 		- rechts
		%   p{2cm} 	- Paragraph mit fester Breite
		%
		% Spaltenbegrenzung mittels der Pipe: 	|
		% Spaltentrenner druch:					&
		% Neue Zeile:							\\
		% Horizontale Linie						\hline
		% ---------------------------
	
		\begin{tabular}{l|c|r}
			linke Spalte & mittlere Spalte & rechte Spalte \\\hline
			1            &        2        &             3 \\
			aaa          &       bbb       &           ccc
		\end{tabular}
		\captionof{table}{Tabelle ohne Gleitumgebung}
		
	\section{Tabelle mit Gleitumgebung}
		% ---------------------------
		% Tabelle in Gleitumgebung
		% ---------------------------
		% Tabelle wird in Gleitumgebung zur automatischen Positionierung eingebunden
		% Angabe der Postion (optional):
		% 	h - hier
		%   t - top
		%	b - bottom
		% 
		% Tabelle wird zentriert und mit Tabellenunterschrift dargestellt
		% Verweis auf Tabelle durch \label{beliebiger Name} und \ref{Name wir bei label}
		% ---------------------------
		
		\begin{table}[htb]													% Gleitumgebung für Tabellen
			\centering															% Tabelle zentrieren
			\begin{tabular}{l|c|r}												% Tabelle erstellen
				linke Spalte & mittlere Spalte & rechte Spalte \\\hline
				1            &        2        &             3 \\
				aaa          &       bbb       &           ccc
			\end{tabular}
			\caption{Ich bin eine Tabelle, die im Tabellenverzeichnis steht}	% Tabellenunterschrift erzeugen
			\label{tab:table}													% Verweis erstellen
		\end{table}
		
	\section{Tabellen mit \texttt{tabularx}}
		% ---------------------------
		% Schöne Tabellen erstellen
		% ---------------------------
		% Verwendung von \usepackage{tabularx} und \usepackage{booktabs}
		% Neue Umgebung \begin{tabularx}{width}{cols}
		% Breite der Tabelle kann angepasst werden
		% Horizontale Linien sehen besser aus
		% Neuer Spaltentyp X - automatische Breite
		% ---------------------------
		
		\begin{table}[htb]
			\centering
			\begin{tabularx}{\linewidth}{lXr}
				\toprule
				Linke Spalte & Mittlere Spalte & Rechte Spalte \\ \midrule
				1            & 2               & 3             \\
				aaa          & bbb             & ccc           \\ \bottomrule
			\end{tabularx}
			\caption{Tabelle mit tabularx}
		\end{table}
	
	% ---------------------------
	% Komplexe Tabellen erstellen
	% ---------------------------
	% Zusammenfassen von Zellen (horizontal) mittels \multicolumn{cols}{pos}{text}
	% Zusammenfassen von Zellen (vertical) mittels \multirow{number of rows}{*}{text}
	% (Package multirow notwendig!)
	% Verfeinerung der Darstellung
	% ---------------------------
	
	\section{Komplexe Tabellen}
		\begin{table}[htb]
			\centering
			\caption{Ich bin eine Tabellenüberschrift}
			\begin{tabularx}{.8\linewidth}{@{}>{\centering}XXX@{}}
				\toprule
				\multicolumn{3}{c}{Überschrift}\\\midrule
				$\sin$ & $\cos$ & Fake\\\cmidrule(r){1-1}\cmidrule(lr){2-2}\cmidrule(l){3-3}
				\multirow{3}{*}{Irgendwas} & was auch immer & $\pi$\\
				& was auch immer & $\pi$\\
				& was auch immer & $\pi$\\\bottomrule
			\end{tabularx}
			\label{tab:tabularx}
		\end{table}