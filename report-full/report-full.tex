% !TeX encoding     = utf8
% !TeX program      = pdflatex
% !TeX spellcheck   = de-DE
% !BIB = biber

% ---------------------------
% Template: Report
% Author:   Felix Faltin
% Version:  1.1.2
% Email:    felix@clickandtry.de
% ---------------------------

\RequirePackage[l2tabu, orthodox]{nag}          % Sucht nach Fehlern und gibt Warnung aus (.log)
%\RequirePackage{fixltx2e}                      % Nicht mehr notwendig

% ---------------------------
% Dokumentklasse
% ---------------------------
% Mögliche Optionen:
%   - parskip = half; full, none
%   - Schriftgröße (11pt [default], 12pt)
%   - Papiergröße (a4paper [default], a5paper)
%   - zweiseitig (twoside, oneside [default])
% ---------------------------

\documentclass[
    12pt,                                       % Schriftgröße
    a4paper,                                    % Papiergröße
    parskip = half,                             % Definiert Absätze
    twoside,                                    % Zweiseitiger Druck
    listof  = totoc,                            % Bilder- und Tab.-verzeichnis im Inhalsverzeichnis
    bibliography = totoc,                       % Quellenverzeichnis im Inhaltszeichnis
    version = last                              % Version
]{scrreprt}                                     % Komascript (vorzugsweise zu verwenden)

% ---------------------------
% Einbinden von Packages
% ---------------------------
% Wichtiger Hinweis für die Kodierung:
%     Inputenc gibt die Kodierung des Dokoments an,
%     die muss mit der Kodierung der Datei übereinstimmen.
%
% Mögliche Einstellungen:
%    Editor       |    inputenc
%    -------------|------------
%    UTF-8        |    utf8        (vorzugsweise verwenden)
%    ISO-8859-1   |    ansinew
%    windows-1252 |    ansinew
%    Apple        |    applemac    (von mir nicht geprüft)
%    Linux        |    latin1
% ---------------------------

% --- Kodierung, Sprache ---
\usepackage[utf8]{inputenc}                     % Kodierung des Dokuments
\usepackage[T1]{fontenc}                        % Schriftzeichen, Umlaute, Silbentrennung
\usepackage[ngerman]{babel}                     % Neue deutsche Rechtschreibung

% --- Schrift ---
\usepackage{lmodern}                            % Moderne Schriftart
\usepackage{microtype}                          % Randausgleich und Neuberechnung der Lücken

% ---------------------------
% Fakulative Packages
% ---------------------------
% Zum Aktivieren Kommentar entfernen
%
%        ---WICHTIG---
% Reihenfolge von einigen Pakete muss beachtet werden,
% da es ansonsten zu "Option clash" kommen kann!
% ---------------------------

% --- Hacks ---
\usepackage{scrhack}                            % Hacks für setspaces, float, hyperref und listings

% --- Tabellen ---
\usepackage{tabularx}                           % Verwendung der Tabularx-Umgebung für Tabellen
\usepackage{booktabs}                           % Übersichtlichere Darstellung von Tabellen
\usepackage{multirow}                           % Zusammenfassen von Tabellenreihen (vertical)

% --- Grafiken ---
\usepackage{tikz}                               % Erstellen von eigenen Grafiken
%\usepackage{float}                             % U.a neue Option H für figure und table

% --- Formatierungen und Design ---
\usepackage{setspace}                           % Ein-, anderhalb- oder zweizeilig
\usepackage[
    % --- Linienart ---                         % Einstellung von Dicke und Breite
%    headtopline,                               % durch name = Dicke:Breite
%    plainheadtopline,
    headsepline,
%    plainheadsepline,
    footsepline,
    plainfootsepline,
%    footbotline,
%    plainfootbotline,
%    ilines,
    clines,
%    olines,
    automark,
%    autooneside = false,                       % ignore optional argument in automark at oneside
]{scrlayer-scrpage}                             % Kopf- und Fußzeile
\usepackage[
    top           = 3cm,                        % Seitenabstand oben
    right         = 2.5cm,                      % Seitenabstand recht
    bottom        = 3cm,                        % Seitenabstand unten
    left          = 2.5cm,                      % Seitenabstand links
    bindingoffset = 1cm,                        % Bindekorrektur
    includeheadfoot
]{geometry}
%\usepackage[
%    table,
%    dvipsnames                                  % Vordefinierte Farben
%]{xcolor}                                       % Farbige Darstellungen (von tikz geladen)
%\usepackage{caption}                           % Bilder- u. Tabellenbeschriftung

% --- Mathematik ---
%\usepackage{cancel}                            % Durchstreichen von Gleichungen
\usepackage{siunitx}                            % Darstellung von Maßzahlen und Maßeinheiten
\usepackage[
    intlimits,                                  % Grenzen Integral
%    %leqno,                                     % Nummerierung links
%    %reqno,                                     % Nummerierung rechts [default]
%    %fleqn                                      % Formel linksbündig mit festem Abstand
]{amsmath}                                      % Erweiterung des Mathemodus
\usepackage[
    all,
    warning,
%    error                                      % Kompilieren von pdfLaTeX kann abgebrochen werden
]{onlyamsmath}                                  % Anzeigen von Fehlern bei Verwendung von amsmath
\usepackage[
    fixamsmath,                                % [default]
    disallowspaces                             % Fehlerbehebung
]{mathtools}                                   % Erweitung zu amsmath

% --- PDF ---
%\usepackage{epstopdf}                          % Konvertiert .eps zu .pdf on-the-fly (graphicx notwendig!)
%\usepackage{pdfpages}                          % PDF Seiten einbinden

% --- Quellenangabe
\usepackage[autostyle]{csquotes}                % Anführungszeichen für deutsche Sprache und Quellenverzeichnis
\usepackage[
    backend      = biber,                       % bibtex oder biber [default]
    style        = numeric-comp,                % Zitierstil
    sorting      = nty,                         % Sortierung (Name, Titel, Jahr)
    natbib       = true,                        % Kompatibilität mit Natbib-Bibliothek
    block        = space,                       % kleiner horizontaler Platz zwischen den Feldern
    backrefstyle = three+,                      % fasst Seiten zusammen, z.B. S. 2f, 6ff, 7-10
    date         = short,                       % Datumsformat
    bibwarn      = true,                        %
    texencoding  = auto,                        % auto-detect the input encoding
    bibencoding  = auto                         % (auto (equal to tex), <encoding>)
]{biblatex}
%\usepackage[
%    square,
%    numbers
%]{natbib}                                      % Zitierstile (NUR für bibtex!)

% --- Zusatzpakete ---
\usepackage{blindtext}                          % Ermöglichst Benutzung von Blindtext

% --- Verweise ---
\usepackage[hidelinks]{hyperref}                % Ermöglicht Verlinkungen [Immer als letztes!], als alternative Option kann "colorlinks" geladen werden
\usepackage[
    ngerman,
    noabbrev,
    nameinlink
]{cleveref}                                     % Erweitert Angabe von Referenzen

% ---------------------------
% Einstellungen von Optionen
% ---------------------------

% --- mathtools ---
%\mathtoolsset{showonlyrefs}                    % Nummierung von Gleichungen nur, wenn diese einen Verweis besitzen

% --- siunitx ---
\sisetup{
%    mode = math, % text is printed using a math font
    detect-all,
    separate-uncertainty  = true,               % Plus-Minus-Zeichen für Fehler
    exponent-product      = \cdot,              % Malzeichen für Exponent
    number-unit-separator = \text{\,},          % Komma als Dezimalzeichen statt Punkt
    output-decimal-marker = {\text{,}},
    per-mode              = fraction            % Darstellung von Brüchen
}

% --- caption ---
%\captionsetup{}

% --- scrlayer-scrpage ---
%\pagestyle{scrheadings}
\clearpairofpagestyles                          % Leeren von Kopf- und Fußzeile

%\ohead{\pagemark}                              % Kopfzeile außen: Seitenzahl
\ihead{\headmark}                               % Kopfzeile innen: chapter und section Titel
\cfoot[-~\pagemark~-]{-~\pagemark~-}            % Fußzeile mitte: Seitenzahl
\automark[section]{chapter}
%\setkomafont{pageheadfoot}{}                   % Änderung von Kopf- und Fußzeile
%\setkomafont{pagenumber}{}
%\setkomafont{headsepline}{\color{red}}

% --- hyperref ---
\hypersetup{
    pdftitle           = {Titel},               % Name des Dokuments
    pdfsubject         = {Einführung},          % Thema der Dokuments
    pdfauthor          = {Felix},               % Autor
%    pdfkeywords       = {},
%    pdfcreator        = {},
%    pdfproducer       = {},
    pdftoolbar         = true,
    pdfmenubar         = false,
%    bookmarks         = true,
    bookmarksopen      = false,                 % bookmarksopen ODER bookmarksopenlevel!
%    bookmarksopenlevel = section,
}

% ---------------------------
\KOMAoptions{
%    headlines = 2                              % Für mehrzeilige Überschriften
%    headinclude,                               % Überschrift in Satzspiegel berücksichtigen
%    BCOR = 1cm,                                % Bindekorrektur
%    DIV = calc                                 % Satzspiegel neuberechnen
}

% ---------------------------
% Allgemeine Einstellungen
% ---------------------------

% --- Kopf- und Fußzeile ---
% Nur ohne scrlayer-scrpages!
%\pagestyle{headings}                                       % Möglichkeiten: plain, headings, empty
%\addtokomafont{pageheadfoot}{\linespread{1}\selectfont}    % Einzeilige Kopf- und Fußzeile

% --- Stil der Seitenzahlen ----
\pagenumbering{Roman}                           % Möglichkeiten: arabic [default], Roman, Alpha, gobble (keine Nummerierung)

% --- Angabe des Speicherorts von Bildern ---
\graphicspath{{img/}}                           % Nur in Verbindung mit graphicx!
% \graphicspath{{folder/subfolder/},{folder2/}}

% --- Zeilenabstand ---
\onehalfspacing                                 % \singlespacing, \onehalfspacing, \doublespacing (Nur mit setspaces!)

% ---------------------------
% Quellenverzeichnis
% ---------------------------
\addbibresource{./bib/referenzen.bib}           % BiBLaTeX, biber (Konfiguration im TeXstudio beachten)
%\bibliography{bib/referenzen}                  % Veraltet (bibtex)

% ---------------------------
% Beginn des Dokuments
% ---------------------------

\begin{document}

    % ---------------------------
    % Einbinden eines Titelblattes
    % (Standardformatierung)
    % ---------------------------

    \title{Bausteine \LaTeX}                    % Titel der Arbeit
    \subtitle{Zusammenfassung von Elementen}    % Untertitel
    \author{Felix Faltin}                       % Autor
    \publishers{Click'n Try}                    % Herausgeber
    \date{\today}                               % Datum
    %    \extratitle{Ich bin eine Schutzseite}  % Schutztitel
    %    \dedication{Ich bin der Beste}         % Widmung

    % ---------------------------
    % Erzeugen von Verzeichnissen
    % ---------------------------
    \maketitle                                  % Titelseite erzeugen
    \tableofcontents                            % Inhaltsverzeichnis
    \listoffigures                              % Bilderverzeichnis
    \listoftables                               % Tabellenverzeichnis

    % --- Umschalten der Nummerierung ---
    \cleardoublepage
    \pagenumbering{arabic}

    % ---------------------------
    % Kapitel und Abschnitte
    % ---------------------------
    % ---------------------------
% Inhalt einfügen
% ---------------------------
% 1. Kapitel (\chapter{title})
%    1.1 Abschnitte (\section{title})
%        1.1.1 Unterabschnitte (\subsection{})
%
% \blindtext benötigt \usepackage{blindtext}
% ---------------------------

%kapitel einfügen
\chapter{Kapitel einfügen}
	% Abschnitt einfügen
    \section{Unterpunkt 1}
	\label{sec:unterpunkt1}
                    
        % Unterpunkt einfügen
        \subsection{Unterpunkt 1}
            \blindtext							% Fügt Text ein, der als Platzhalter dient
        
        % Unterpunkt einfügen    
        \subsection{Unterpunkt 2}
            \blindtext							% Kann mit beliebigen Text ersetzt werden
    
    \section{Unterpunkt 2}
    
	    % Unterpunkt einfügen
	    \subsection{Unterpunkt 1}
	    \label{subsec:lalalulale}
	    \blindtext								% Fügt Text ein, der als Platzhalter dient
	    
	    % Unterpunkt einfügen    
	    \subsection{Unterpunkt 2}
	    \blindtext								% Kann mit beliebigen Text ersetzt werden

    % ---------------------------
    % Verweise einfügen
    % ---------------------------
    \chapter{Verweise}
    \section{Verweise erzeugen}
        % ---------------------------
        % Verweise erzeugen
        % ---------------------------
        % \label{key} erstellt Verweis
        % \ref{key} verweist auf \label{key}
        %
        % Benennung der Verweise (best practice):
        %     - Verwendung der Umgebung als Prafix
        %    - keine Leerzeichen
        %    - keine Umlaute
        % ---------------------------
        Im Kapitel \ref{sec:unterpunkt1} werden Abschnitte erklärt. In Unterkapitel \ref{subsec:lalalulale} steht Blindtext.

    \section{Präfixe}
        % ---------------------------
        % Verweise erzeugen
        % ---------------------------
        % Die Schlüsselwörter können frei gewählt werden.
        % Zur besseren Übersicht empfiehlt sich folgender Kodex
        % xxxxx mit jeweiligen Schlüsselwort ersetzen
        % ---------------------------
        \begin{itemize}
            \item Bilder mit figure-Umgebung: \textbf{fig:xxxxx}
            \item Tabellen mit table-Umgebung: \textbf{fig:xxxxx}
            \item Kapitel: \textbf{chap:xxxxx}
            \item Abschnitte: \textbf{sec:xxxxx} \dots \textbf{subsec:xxxxx}
            \item Quellenangabe: \cite{papula2009}
        \end{itemize}


    % ---------------------------
    % Aufzählungen einfügen
    % ---------------------------
    % ---------------------------
% Einfache Aufzählungen
% ---------------------------
% Aufzählungen können mit
%     - \begin{itemize}     -> unsortierte Liste
%    - \begin{enumerate}    -> sortiere Liste (nummeriert)
% ---------------------------

\chapter{Aufzählungen}
    % ---------------------------
    % Unsortierte Liste
    % ---------------------------
    \begin{itemize}                                % Umgebung itemize (Auflistung)
        \item Beispiel                            % Stichpunkt hinzufügen
        \item Example
        \item Hallo Welt
    \end{itemize}                                % Umgebung beenden

    % ---------------------------
    % Sortierte Liste
    % ---------------------------
    \begin{enumerate}                            % Umgebung enumerate (Aufzählung)
        \item Ich bin erstens                    % Stichpunkt hinzufügen
        \item Nach dir kommt zweitens
        \item Aller guten Dinge sind drei
    \end{enumerate}                                % Umgebung beenden

    % ---------------------------
    % Kombiniert und verschachtelt
    % Die Umgebungen können beliebig kombiniert
    % und verschachtelt werden
    % ---------------------------
    \begin{enumerate}
        \item Personen
        \begin{itemize}
            \item Minion
            \item Dinos
            \item Nessi
        \end{itemize}
        \item Gebäude
        \begin{enumerate}
            \item Wohnungen
            \begin{itemize}
                \item Haus
                \item Modell
            \end{itemize}
            \item Schlafen
            \begin{itemize}
                \item Bett
                \item Hotel
            \end{itemize}
        \end{enumerate}
    \end{enumerate}

    % ---------------------------
    % Bilder einfügen
    % ---------------------------
    % ---------------------------
% Darstellung von Bildern
% ---------------------------
% Package graphicx notwenig
% 
% HINWEIS:
% Bei Verwendung von Gleitumgebungen kann es anfangs zu 
% ungewollten Darstellungen kommen!
% Es sollte IMMER erst der Text vollständig eingefügt werden,
% und zum Schluss die Formatierung bearbeitet werden!
% ---------------------------
\chapter{Bilder}
	
	\section{Grundbefehl}
		% ---------------------------
		% Grundbefehl zum Einbinden von 
		% ---------------------------
		% \includegraphics[options]{filename} ohne Endung (.jpg, .png)
		% Optionen:	width 	- Breitenangabe (cm, mm, % oder \linewidth)
		%			height	- Höhenangabe (automatisch bei width-Angabe)
		%			angle	- Drehung hinzufügen
		% ---------------------------
		\includegraphics{texstudio-encoding}
	
	\section{Gleitumgebung für Bilder}
		% ---------------------------
		% Bilder in einer Gleitumgebung
		% ---------------------------
		% Gleitumgebung für Bilder - figure
		% Optionen h - here, t - top, b - bottom
		% ---------------------------
		\begin{figure}[htb]									% Gleitumgebung
			\centering										% Bild zentrieren
			\includegraphics[width=.75\linewidth]{texstudio-encoding}
			\caption{Bildunterschrift mit Gleitumgebung}	% Bildunterschrift einfügen
			\label{fig:figure}								% Verweis einfügen
		\end{figure}
		
		\begin{figure}[htb]
			\centering
			\includegraphics[width=.5\linewidth, angle=45]{texstudio-encoding}
			\caption{Bildunterschrift mit Gleitumgebung}
			\label{fig:figure2}
		\end{figure}
		
		\clearpage											% Einfügen aller Grafiken
	
	\section{Bild ohne Gleitumgebung}
		% ---------------------------
		% Bilder ohne Gleitumgebung
		% ---------------------------
		% Bildunterschrift nur mit Gleitumgebung
		% Alternativ: \captionof{float type}{heading}
		% mit Package caption
		% ---------------------------
		\includegraphics[width=.95\textwidth]{texstudio-config}
		\captionof{figure}{Bildunterschrift ohne Gleitumgebung}

    % ---------------------------
    % Tabellen einfügen
    % ---------------------------
    \chapter{Tabellen}
		
	\section{Einfache Tabellen ohne Pakete}
		% ---------------------------
		% Einfache Tabellen darstellen
		% ---------------------------
		% Spaltenangabe durch
		%   l 		- links
		%   c 		- zentriert
		%   r 		- rechts
		%   p{2cm} 	- Paragraph mit fester Breite
		%
		% Spaltenbegrenzung mittels der Pipe: 	|
		% Spaltentrenner druch:					&
		% Neue Zeile:							\\
		% Horizontale Linie						\hline
		% ---------------------------
	
		\begin{tabular}{l|c|r}
			linke Spalte & mittlere Spalte & rechte Spalte \\\hline
			1            &        2        &             3 \\
			aaa          &       bbb       &           ccc
		\end{tabular}
		\captionof{table}{Tabelle ohne Gleitumgebung}
		
	\section{Tabelle mit Gleitumgebung}
		% ---------------------------
		% Tabelle in Gleitumgebung
		% ---------------------------
		% Tabelle wird in Gleitumgebung zur automatischen Positionierung eingebunden
		% Angabe der Postion (optional):
		% 	h - hier
		%   t - top
		%	b - bottom
		% 
		% Tabelle wird zentriert und mit Tabellenunterschrift dargestellt
		% Verweis auf Tabelle durch \label{beliebiger Name} und \ref{Name wir bei label}
		% ---------------------------
		
		\begin{table}[htb]													% Gleitumgebung für Tabellen
			\centering															% Tabelle zentrieren
			\begin{tabular}{l|c|r}												% Tabelle erstellen
				linke Spalte & mittlere Spalte & rechte Spalte \\\hline
				1            &        2        &             3 \\
				aaa          &       bbb       &           ccc
			\end{tabular}
			\caption{Ich bin eine Tabelle, die im Tabellenverzeichnis steht}	% Tabellenunterschrift erzeugen
			\label{tab:table}													% Verweis erstellen
		\end{table}
		
	\section{Tabellen mit \texttt{tabularx}}
		% ---------------------------
		% Schöne Tabellen erstellen
		% ---------------------------
		% Verwendung von \usepackage{tabularx} und \usepackage{booktabs}
		% Neue Umgebung \begin{tabularx}{width}{cols}
		% Breite der Tabelle kann angepasst werden
		% Horizontale Linien sehen besser aus
		% Neuer Spaltentyp X - automatische Breite
		% ---------------------------
		
		\begin{table}[htb]
			\centering
			\begin{tabularx}{\linewidth}{lXr}
				\toprule
				Linke Spalte & Mittlere Spalte & Rechte Spalte \\ \midrule
				1            & 2               & 3             \\
				aaa          & bbb             & ccc           \\ \bottomrule
			\end{tabularx}
			\caption{Tabelle mit tabularx}
		\end{table}
	
	% ---------------------------
	% Komplexe Tabellen erstellen
	% ---------------------------
	% Zusammenfassen von Zellen (horizontal) mittels \multicolumn{cols}{pos}{text}
	% Zusammenfassen von Zellen (vertical) mittels \multirow{number of rows}{*}{text}
	% (Package multirow notwendig!)
	% Verfeinerung der Darstellung
	% ---------------------------
	
	\section{Komplexe Tabellen}
		\begin{table}[htb]
			\centering
			\caption{Ich bin eine Tabellenüberschrift}
			\begin{tabularx}{.8\linewidth}{@{}>{\centering}XXX@{}}
				\toprule
				\multicolumn{3}{c}{Überschrift}\\\midrule
				$\sin$ & $\cos$ & Fake\\\cmidrule(r){1-1}\cmidrule(lr){2-2}\cmidrule(l){3-3}
				\multirow{3}{*}{Irgendwas} & was auch immer & $\pi$\\
				& was auch immer & $\pi$\\
				& was auch immer & $\pi$\\\bottomrule
			\end{tabularx}
			\label{tab:tabularx}
		\end{table}

    % ---------------------------
    % Mathematische Formeln
    % ---------------------------
    % ---------------------------
% Darstellung von Formeln
% ---------------------------
% math             -> inline Mathemodus
% displaymath     -> Mathemodus mit einzelner Zeile
% equation         -> Mathemodus mit einzelner Zeile und Nummerierung
% ---------------------------
\chapter{Mathematische Formeln}
    \section{Formeln}
        \SI{2,4e4}{\gram \per \mole}

    \section{Einfache Formeln}
        % ---------------------------
        % Einfacher Mathemodus inline
        % ---------------------------
        \begin{math}
            A = \dfrac{\pi}{4}\,d^2
        \end{math}
        \newline
        % ---------------------------
        % Kurzschreibweise von math
        % Beachte l2tabu
        % ftp://ftp.dante.de/tex-archive/info/german/l2tabu/l2tabu.pdf
        % ---------------------------
        \(A = \dfrac{\pi}{4}\,d^2\)

        % ---------------------------
        % Einfacher Mathemodus OHNE Nummerierung
        % equation* ist zu bevorzugen bei Verwendung von amsmath
        % ---------------------------
        Satz des Pythagoras
        \begin{displaymath}
            c^2 = a^2 + b^3
        \end{displaymath}

        % ---------------------------
        % Kurzschreibweise von displaymath
        % Beachte l2tabu
        % ftp://ftp.dante.de/tex-archive/info/german/l2tabu/l2tabu.pdf
        % ---------------------------
        Alternativ:
        \[
            a = \sqrt{c^2 - b^2}
        \]

    \section{Formeln mit Nummerierung}
        % ---------------------------
        % Mathemodus MIT Nummerierung
        % ---------------------------
        Volumenänderungsarbeit
        \begin{equation}
            W_{V12} = - \int_{V_1}^{V_2} p \cdot dV + W_{r12}
        \end{equation}

        \begin{equation}
            W_{V12} = - \int\limits_{V_1}^{V_2} p \cdot dV + W_{r12}
        \end{equation}

    \section{Geordnete Formeln}

        % ---------------------------
        % Mathemodus mit Ausrichtung
        % ---------------------------
        % Ausrichtung erfolgt mithilfe des Trenners "&"
        % Zeilenende durch \\ kennzeichnen
        % Keine Nummerierung durch \nonumber\\
        % ---------------------------
        \begin{align}
            t_u &= \sin(x) + \cos^2(x)\\
            \sigma_{out} &= \dfrac{1}{2} \exp^{b_2 - b_1}\nonumber\\
            \varrho_{12} &= \left(\dfrac{a \cdot b}{\pi}\right) \cdot \sqrt[4]{\pi}
        \end{align}

    % ---------------------------
    % Quellenverzeichnis
    % ---------------------------
%    \bibliographystyle{abbrvnat}
%    \bibliography{./bib/referenzen}                % BibTeX

    \printbibliography[title=Literaturverzeichnis]  % BiBLaTeX, Biber
\end{document}
