\chapter{Verweise}
    \section{Verweise erzeugen}
        % ---------------------------
        % Verweise erzeugen
        % ---------------------------
        % \label{key} erstellt Verweis
        % \ref{key} verweist auf \label{key}
        %
        % Benennung der Verweise (best practice):
        %     - Verwendung der Umgebung als Prafix
        %    - keine Leerzeichen
        %    - keine Umlaute
        % ---------------------------
        Im Kapitel \ref{sec:unterpunkt1} werden Abschnitte erklärt. In Unterkapitel \ref{subsec:lalalulale} steht Blindtext.

    \section{Präfixe}
        % ---------------------------
        % Verweise erzeugen
        % ---------------------------
        % Die Schlüsselwörter können frei gewählt werden.
        % Zur besseren Übersicht empfiehlt sich folgender Kodex
        % xxxxx mit jeweiligen Schlüsselwort ersetzen
        % ---------------------------
        \begin{itemize}
            \item Bilder mit figure-Umgebung: \textbf{fig:xxxxx}
            \item Tabellen mit table-Umgebung: \textbf{fig:xxxxx}
            \item Kapitel: \textbf{chap:xxxxx}
            \item Abschnitte: \textbf{sec:xxxxx} \dots \textbf{subsec:xxxxx}
            \item Quellenangabe: \cite{papula2009}
        \end{itemize}
