% ---------------------------
% Darstellung von Bildern
% ---------------------------
% Package graphicx notwenig
% 
% HINWEIS:
% Bei Verwendung von Gleitumgebungen kann es anfangs zu 
% ungewollten Darstellungen kommen!
% Es sollte IMMER erst der Text vollständig eingefügt werden,
% und zum Schluss die Formatierung bearbeitet werden!
% ---------------------------
\chapter{Bilder}
	
	\section{Grundbefehl}
		% ---------------------------
		% Grundbefehl zum Einbinden von 
		% ---------------------------
		% \includegraphics[options]{filename} ohne Endung (.jpg, .png)
		% Optionen:	width 	- Breitenangabe (cm, mm, % oder \linewidth)
		%			height	- Höhenangabe (automatisch bei width-Angabe)
		%			angle	- Drehung hinzufügen
		% ---------------------------
		\includegraphics{texstudio-encoding}
	
	\section{Gleitumgebung für Bilder}
		% ---------------------------
		% Bilder in einer Gleitumgebung
		% ---------------------------
		% Gleitumgebung für Bilder - figure
		% Optionen h - here, t - top, b - bottom
		% ---------------------------
		\begin{figure}[htb]									% Gleitumgebung
			\centering										% Bild zentrieren
			\includegraphics[width=.75\linewidth]{texstudio-encoding}
			\caption{Bildunterschrift mit Gleitumgebung}	% Bildunterschrift einfügen
			\label{fig:figure}								% Verweis einfügen
		\end{figure}
		
		\begin{figure}[htb]
			\centering
			\includegraphics[width=.5\linewidth, angle=45]{texstudio-encoding}
			\caption{Bildunterschrift mit Gleitumgebung}
			\label{fig:figure2}
		\end{figure}
		
		\clearpage											% Einfügen aller Grafiken
	
	\section{Bild ohne Gleitumgebung}
		% ---------------------------
		% Bilder ohne Gleitumgebung
		% ---------------------------
		% Bildunterschrift nur mit Gleitumgebung
		% Alternativ: \captionof{float type}{heading}
		% mit Package caption
		% ---------------------------
		\includegraphics[width=.95\textwidth]{texstudio-config}
		\captionof{figure}{Bildunterschrift ohne Gleitumgebung}